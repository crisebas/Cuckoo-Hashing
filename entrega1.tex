\documentclass[final,a4paper,twocolumn,romanappendices]{IEEEtran}
\usepackage[utf8]{inputenc}
\usepackage[T1]{fontenc} %CUando uno descargue el PDF sea posible 
\usepackage[spanish]{babel}
\spanishdatedel %para el "del" de la fecha
\usepackage{amsmath}
\usepackage[cmintegrals]{newtxmath}
\usepackage{bm}
\usepackage{hyperref}
\usepackage{graphicx}%[demo]

\title{Implementación del hash del cuco en firmas digitales}
\author{Anthony Bautista$^{1}$, Cristopher García$^{2}$, Rosa Limachi$^{3}$ y Julio Rosales$^{4}$\\
\small{$^{1}$ $^{2}$ $^{3}$ Ciencias de la Computación $^{4}$ Matemática\\}
\small{Universidad Nacional de Ingeniería, Lima, Perú.\\}
\small{\texttt{\{$^{1}$abautistal, $^{3}$rlimachip\}@uni.pe \{$^{2}$crisebas100, $^{4}$julio845\}@gmail.com}}
}

\begin{document}
\maketitle

\begin{abstract}
Este proyecto se basa en la creación de una firma digital implementando el hash del cuco para la “solución” de las colisiones inherentes a una función hash. Además, se usará el criptosistema RSA para la generación de claves tanto públicas como privadas. Se utilizará el lenguaje Python para la programación de los algoritmos. 






\end{abstract}

\section{Introducción}

Cuando leemos un texto o mensaje cualquiera, generalmente pasamos por alto algunos detalles que nos debería importar, por ejemplo ¿El mensaje lo envió la persona que creemos?, ¿El mensaje fue alterado desde su emisión?.
\\
Las firmas digitales nos aseguran que el emisor del mensaje es de la persona original y que no fue alterado por un tercero, con una serie algoritmos que requieren de funciones hash, y de una técnica de hashing.
\\
Implementaremos el hash del cuco, que es un método que “resuelve” las colisiones hash creando como mínimo dos funciones hash, lo que da la posibilidad de ubicar el valor en otra tabla hash.
\\
Nuestro objetivo principal es generar firmas digitales implementando el hash del cuco.

\section{Preliminares}
\subsection{Función Hash}
Función que a partir de una entrada que suele ser una cadena, genera una salida (cadena) de longitud fija, esta tiene información.\\
La colisión hash se produce cuando entradas distintas generan el mismo valor en una función hash.

\subsection{Tabla hash}
Estructura de datos que relaciona claves y valores para cada elemento que guarda. Utilizaremos una función hash para transformar la clave de un dato e identificar el lugar que ocupará en la tabla.

\subsection{Firma digital}
certificado digital que emite la entidad para garantizar la autenticidad de la información, evitando el repudio del mensaje. Para lograr esto se toma el mensaje que se desea enviar, se le aplica una función hash y al resultado se le aplica la clave privada del usuario que envía el mensaje. Luego, el destinatario aplica una clave privada al resultado anterior y compara con el hash obtenido del mensaje recibido. Si ambos resultados son iguales, significa que el mensaje es auténtico.\\

\section{Estado del Arte}

\begin{enumerate}

\item
Encriptación RSA de Archivos de texto: Explica conceptos como clave pública, clave privada, encriptación y descifrado de mensaje. Además, muestra el algoritmo RSA y una aplicación en lenguaje java.$[1]$

\item
Algoritmo hash y vulnerabilidad a ataques: Explica el problemas del ciframiento de los datos al aplicar encriptación por hash, donde se propone como solución emplear dos algoritmos como el SHA-1 y RIPEND-160.$[2]$
\end{enumerate}


\section{Diseño del experimento}
%metodo
Para obtener el resultado, utilizaremos el lenguaje Python.
Usaremos listas que serán nuestras tablas hash y por medio de la programación orientada a objetos podremos instanciar las tablas que nos permita implementar el hash del cuco.



\end{document}
